% Config
\documentclass{beamer}
\usepackage[utf8]{inputenc}

% Packages
\usepackage{multicol}
\usepackage[francais]{babel}
\usepackage{amsfonts, amsthm, amssymb, amsmath}
\usepackage{xcolor}
\usepackage{mathtools}
\usepackage{ninecolors}
\usepackage{physics}
\usepackage{hyperref}
\usepackage{adjustbox}
\usepackage{listings}

\hypersetup{colorlinks=true, linkcolor=black, urlcolor=black}

\title{TP5 Thermodynamique}
\author{BERREDO DE LA COLINA Lucas\\ MARTIN Lola}
\date{}

\usetheme[nofirafonts]{focus}

\lstset{frame=tb,
  language=Python,
  aboveskip=3mm,
  belowskip=3mm,
  showstringspaces=false,
  columns=flexible,
  basicstyle={\small\ttfamily},
  numbers=none,
  numberstyle=\tiny\color{gray},
  keywordstyle=\color{blue},
  commentstyle=\color{darkgreen},
  stringstyle=\color{mauve},
  breaklines=true,
  breakatwhitespace=true,
  tabsize=4
}

% Commands
\newcommand{\newlines}{\newline\newline}



% Document
\begin{document}




\begin{frame}

\Huge{TP5 Thermodynamique}

\Large{Partie 1: Rayonnement}
\\[2em]
\large{BERREDO DE LA COLINA Lucas\\ MARTIN Lola}

\end{frame}





\begin{frame}
\frametitle{Rappel théorique}

\begin{itemize}
	\item{Corps noir / corps gris}
	\item{Radiation dans le vide}

\end{itemize}

\end{frame}





\begin{frame}
\frametitle{Explication théorique}

\begin{itemize}
	\item{Analogue électrocinétique}
	\item{Résolution exacte}
	\item{Résolution approchée}
	
\end{itemize}

\end{frame}





\begin{frame}
\frametitle{Dispositif expérimental}

\begin{itemize}
	\item{Deux échantillons (gris, noir)\newline}
	\item{Deux chambres:\newline
	\begin{itemize}
		\item{Four ($200^{\circ}C$)\newline}
		\item{Refroidessement à l'eau\newline}
	\end{itemize}}
	\item{Elles peuvent être mises sous vide\newline}
	\item{Mesures de temperature analogiques (100 points, 15 min)}
\end{itemize}
\end{frame}





\begin{frame}
\frametitle{Experiences}

\begin{enumerate}
	\item{{\color{gray7}Corps gris}{\color{gray4}, {\color{red}chauffage}, vide}\newline}
	\item{{\color{gray7}Corps gris}{\color{gray4}, {\color{blue5}refroidissement}, vide}\newline}
	\item{{\color{gray7}Corps gris}{\color{gray4}, {\color{red}chauffage}, sans vide}\newline}
	\item{{\color{gray7}Corps gris}{\color{gray4}, {\color{blue5}refroidissement}, sans vide}\newline}
	\item{{\color{black}Corps noir}{\color{gray4}, {\color{red}chauffage}, vide}\newline}
	\item{{\color{black}Corps noir}{\color{gray4}, {\color{red}chauffage}, sans vide}\newline}

\end{enumerate}
\end{frame}





\begin{frame}
\frametitle{Organisation des données}

\centering
\begin{minipage}{0.48\textwidth}
    \centering
    \includegraphics[width=\linewidth]{Fig/ls-csv.png}
\end{minipage}
\hfill
\begin{minipage}{0.48\textwidth}
    \centering
    \adjustbox{valign=c}{
        \includegraphics[width=\linewidth]{Fig/example-csv.png}}
\end{minipage}
\end{frame}





\begin{frame}
\frametitle{Approximation graphique 1er ordre}

Il ne faut que vérifier les valeur initiales, finales et approcher $\tau$ de façon qu'on trouve des courbes proches

\begin{figure}
\includegraphics[width=0.75\textwidth]{Fig/t1-100-t2-82.png}
\caption{Example representation graphique: Vert: points experimentaux, Bleu: courbe théorique}
\end{figure}

\end{frame}





\begin{frame}
\frametitle{Approximation graphique 1er ordre}

Nous obtenons les prochains valeurs:
\begin{enumerate}
	\item{{\color{gray7}Corps gris}{\color{gray4}, {\color{red}chauffage}, vide} \hfill $\tau = $\hspace{4em} \newline}
	\item{{\color{gray7}Corps gris}{\color{gray4}, {\color{blue5}refroidissement}, vide} \hfill $\tau = $\hspace{4em} \newline}
	\item{{\color{gray7}Corps gris}{\color{gray4}, {\color{red}chauffage}, sans vide} \hfill $\tau = $\hspace{4em} \newline}
	\item{{\color{gray7}Corps gris}{\color{gray4}, {\color{blue5}refroidissement}, sans vide} \hfill $\tau = $\hspace{4em} \newline}
	\item{{\color{black}Corps noir}{\color{gray4}, {\color{red}chauffage}, vide} \hfill $\tau = $\hspace{4em} \newline}
	\item{{\color{black}Corps noir}{\color{gray4}, {\color{red}chauffage}, sans vide}\hfill $\tau = $\hspace{4em} \newline}
\end{enumerate}
\end{frame}





\begin{frame}
\frametitle{Approximation graphique 2ème ordre}
\begin{enumerate}
	\item{{\color{gray7}Corps gris}{\color{gray4}, {\color{red}chauffage}, vide} \hfill $\tau = $\hspace{4em} \newline}
	\item{{\color{gray7}Corps gris}{\color{gray4}, {\color{blue5}refroidissement}, vide} \hfill $\tau = $\hspace{4em} \newline}
	\item{{\color{gray7}Corps gris}{\color{gray4}, {\color{red}chauffage}, sans vide} \hfill $\tau = $\hspace{4em} \newline}
	\item{{\color{gray7}Corps gris}{\color{gray4}, {\color{blue5}refroidissement}, sans vide} \hfill $\tau = $\hspace{4em} \newline}
	\item{{\color{black}Corps noir}{\color{gray4}, {\color{red}chauffage}, vide} \hfill $\tau = $\hspace{4em} \newline}
	\item{{\color{black}Corps noir}{\color{gray4}, {\color{red}chauffage}, sans vide}\hfill $\tau = $\hspace{4em} \newline}
\end{enumerate}

\end{frame}





\begin{frame}
\frametitle{Approximation numérique avec Python}
Comme nous avons la résolution pour $\tau$, nous pouvons donner ça vers un $\texttt{curve\_fit}$ dans Python.

\begin{figure}
\includegraphics[width=\textwidth]{Fig/Python_Refroid.png}
\caption{Exemple refroidissement. Il y a aussi un fichier pour chauffement.}
\end{figure}

\end{frame}






\begin{frame}
\frametitle{Comparaison des résultats}

\begin{table}[htdp]
\begin{center}\begin{tabular}{|c|c|c|c|}
\hline
Expérience & 1er ordre & 2ème ordre & Python \\
\hline
1 & 0 & 0 & 0 \\
2 & 0 & 0 & 0 \\
3 & 0 & 0 & 0 \\
4 & 0 & 0 & 0 \\
5 & 0 & 0 & 0 \\
6 & 0 & 0 & 0\\
\hline
\end{tabular} 
\caption{Valeurs du paramètre $\tau$}
\end{center}
\label{defaulttable}
\end{table}

\end{frame}





\begin{frame}
\frametitle{Explication des résultats}

\end{frame}





\begin{frame}
\frametitle{Approximation numérique avec Python}
Résultats:
\begin{enumerate}
	\item{{\color{gray7}Corps gris}{\color{gray4}, {\color{red}chauffage}, vide} \hfill $\tau = 96.0399...$\hspace{4em} \newline}
	\item{{\color{gray7}Corps gris}{\color{gray4}, {\color{blue5}refroidissement}, vide} \hfill $\tau = 86.1429...$\hspace{4em} \newline}
	\item{{\color{gray7}Corps gris}{\color{gray4}, {\color{red}chauffage}, sans vide} \hfill $\tau = 99.2634...$\hspace{4em} \newline}
	\item{{\color{gray7}Corps gris}{\color{gray4}, {\color{blue5}refroidissement}, sans vide} \hfill $\tau = 84.5618...$\hspace{4em} \newline}
	\item{{\color{black}Corps noir}{\color{gray4}, {\color{red}chauffage}, vide} \hfill $\tau = 111.8591...$\hspace{4em} \newline}
	\item{{\color{black}Corps noir}{\color{gray4}, {\color{red}chauffage}, sans vide}\hfill $\tau = 90.5110...$\hspace{4em} \newline}
\end{enumerate}
\end{frame}





\begin{frame}

\Huge{TP5 Thermodynamique}

\Large{Partie 2: Loi de Stefan}
\\[2em]
\large{BERREDO DE LA COLINA Lucas\\ MARTIN Lola}

\end{frame}





\begin{frame}
\frametitle{Avertissement}
Bien que nous ayons travaillé avec l'équipement et observé des résultats avec {\color{red} Mme Nom}, nous n'avons pas enregistré de résultats numériques.
\end{frame}





\begin{frame}
\frametitle{Rappel théorique}

\begin{itemize}
	\item{Deduction Loi Stefan}

\end{itemize}

\end{frame}





\begin{frame}
\frametitle{Dispositif experimental}

\begin{itemize}
	\item{Deux parties:\newline
	\begin{itemize}
		\item{Côté emmisive - Boule à cuivre ``corps noir''\newline}
		\item{Côté receptive - Thermopile CA2 (filtre en option) et multimètre\newlines}
	\end{itemize}}
	\item{Emmisivité fixé - mesure du puissance avec le multimètre\newline}
	\item{Il faut attendre après chaque changement vers la stabilisation\newline}
	\item{Mesures a plusieurs distances (0,3; 0,4; 0,8; 1,2m) et temperatures (20, 60, 90, 120 $^\circ C$)}
	
\end{itemize}	
	
\end{frame}





\begin{frame}
\frametitle{Approximation des résultats}

Comme rappel, nous n'avons pas fait cette expérience avec tous les points nécessaires. Donc, on fera un approche théoretique, puis une simulation avec l'aide de R, et finalement, nous traiterons ces données au lieu des expérimentales.
	
\begin{itemize}
	\item{Fixons la distance $d$}
	\item{Avec le datasheet, trouvons la quantité qu'on attend comme paramètre}
	\item{Simulation de la v.a. avec bruit normal}
	\item{Analyse des données}

\end{itemize}
\end{frame}


\end{document}