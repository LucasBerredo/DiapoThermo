% Config
\documentclass{beamer}
\usepackage[utf8]{inputenc}

% Packages
\usepackage{multicol}
\usepackage{babel}
\usepackage{amsfonts, amsthm, amssymb, amsmath}
\usepackage{xcolor}
\usepackage{mathtools}
\usepackage{ninecolors}
\usepackage{physics}
\usepackage{hyperref}

\hypersetup{colorlinks=true, linkcolor=black, urlcolor=black}

\title{TP5 Thermodynamique}
\author{BERREDO DE LA COLINA Lucas\\ MARTIN Lola}
\date{}

\usetheme[nofirafonts]{focus}

% Commands
\newcommand{\newlines}{\newline\newline}



% Document
\begin{document}




\begin{frame}

\Huge{TP5 Thermodynamique}

\Large{Partie 1: Rayonnement}
\\[2em]
\large{BERREDO DE LA COLINA Lucas\\ MARTIN Lola}

\end{frame}





\begin{frame}
\frametitle{Rappel théorique}

\begin{itemize}
	\item{Corps noir / corps gris}
	\item{Radiation dans le vide}

\end{itemize}

\end{frame}





\begin{frame}
\frametitle{Explication théorique}

\begin{itemize}
	\item{Analogue électrocinétique}
	\item{Résolution exacte}
	\item{Résolution approchée}
	
\end{itemize}

\end{frame}





\begin{frame}
\frametitle{Dispositif experimental}

\begin{itemize}
	\item{Corps gris}
	\item{Corps noir}

\end{itemize}
\end{frame}





\begin{frame}
\frametitle{Resultats experimentaux - Interpretation}

\begin{itemize}
	\item{Fichiers .csv et Excel}
	\item{Approximation graphique 1er ordre}
	\item{Approximation graphique 2eme ordre}
	\item{Approximation numérique 2eme ordre avec Python}

\end{itemize}
\end{frame}





\begin{frame}

\Huge{TP5 Thermodynamique}

\Large{Partie 2: Loi de Stefan}
\\[2em]
\large{BERREDO DE LA COLINA Lucas\\ MARTIN Lola}

\end{frame}





\begin{frame}
\frametitle{Avertissement}
Bien que nous ayons travaillé avec l'équipement et observé des résultats avec {\color{red} Mme Nom}, nous n'avons pas enregistré de résultats numériques.
\end{frame}





\begin{frame}
\frametitle{Rappel théorique}

\begin{itemize}
	\item{Deduction Loi Stefan}

\end{itemize}

\end{frame}





\begin{frame}
\frametitle{Dispositif experimental}

\begin{itemize}
	\item{Explication dispositif}
	
\end{itemize}	
	
\end{frame}





\begin{frame}
\frametitle{Approximation des résultats}

\begin{itemize}
	\item{Rappel: on n'a pas les résultats, mais on peut approximer}
	\item{Simulation des données}
	\item{Explication du valeur}
	
\end{itemize}
\end{frame}


\end{document}